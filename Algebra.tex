% Please use XelaTeX to compile
\documentclass[a4paper]{book}

\usepackage{amsmath}
\usepackage{amssymb}
\usepackage{array}
\usepackage{extarrows}
\usepackage{enumitem}
\usepackage{float}
\usepackage{fancyhdr}
\usepackage[perpage]{footmisc}
\usepackage{graphicx}
\usepackage{mathtools}
\usepackage{manfnt}
\usepackage{marginnote}
\usepackage{multirow}
\usepackage[thmmarks]{ntheorem}
{
	\theoremheaderfont{\normalfont}
	\theorembodyfont{\normalfont}
	\theoremsymbol{\mbox{$\Box$}}
	\newtheorem*{proof}{\textit{Proof.}}
}
\usepackage{xeCJK}
\usepackage[all, pdf]{xy}

%
\usepackage[margin=5pt]{geometry}
\usepackage[T1]{fontenc}
\usepackage{multicol}
\usepackage{tikz}
\usepackage{xcoffins}
\usepackage{color}
\usepackage{atbegshi}% http://ctan.org/pkg/atbegshi

\geometry{a4paper,scale=0.8,centering}

\newcommand\cbox[2][.8]{{\setlength\fboxsep{0pt}\colorbox[gray]{#1}{#2}}}

\theorembodyfont{\normalfont}
\newtheorem{definition}{Definition}[section]
\newtheorem{axiom}{Axiom}[chapter]
\newtheorem{assumption}{Assumption}[chapter]
\newtheorem{corollary}{Corollary}[section]
\newtheorem{lemma}{Lemma}[section]
\newtheorem{example}{Example}[section]
\newtheorem{remark}{Remark}[section]
\newtheorem{proposition}{Proposition}[section]
\newtheorem*{verify}{Verify}

\newcommand\im{\mathrm{i}}

\begin{document}
		\pagestyle{empty}
	\begin{titlepage} %start
		\newgeometry{margin=5pt}
		\begin{figure*}[h]
			\NewCoffin \result
			\NewCoffin \aaa
			\NewCoffin \bbb
			\NewCoffin \ccc
			\NewCoffin \ddd
			\NewCoffin \eee
			\NewCoffin \fff
			\NewCoffin \rulei
			\NewCoffin \ruleii
			\NewCoffin \ruleiii
			\SetHorizontalCoffin \result {}
			\SetHorizontalCoffin \aaa {\fontsize{52}{50}\sffamily\bfseries\makebox[12cm][c]{Algebra~I}}
			\SetHorizontalCoffin \bbb {\fontsize{40}{50}\sffamily\bfseries My Mathematics Notes}
			\SetHorizontalCoffin \ccc {\fontsize{12}{10}\sffamily 
				\qquad\qquad By A Student from the Australian National University, Canberra \textbullet{} Updated:\,\today}
			\SetHorizontalCoffin \ddd {\fontsize{28}{20}\sffamily Ph\ae drus}
			\SetVerticalCoffin \eee {250pt}
			{\raggedleft\fontsize{35}{50}\sffamily\bfseries 
				\medskip Linear Algebra}
			\SetVerticalCoffin \fff {140pt}
			{\raggedright \fontsize{13}{14}\sffamily\bfseries 
				~\\~\\~\\~\\~\\~\\~\\~\\~}
			\RotateCoffin \bbb {90}
			\RotateCoffin \ccc {270}
			\SetHorizontalCoffin \rulei  {\color{red}\rule{6.5in}{1pc}}
			\SetHorizontalCoffin \ruleii {\color{red}\rule{1pc}{23.5cm}}
			\SetHorizontalCoffin \ruleiii{\color{black}\rule{5pt}{150pt}}
			\JoinCoffins \result                \aaa 
			\JoinCoffins \result[\aaa-t,\aaa-r] \rulei   [b,r](0pt,2mm)
			\JoinCoffins \result[\aaa-b,\aaa-l] \bbb     [B,r](0pt,0pt)
			\JoinCoffins \result[\bbb-t,\bbb-r] \ruleii  [t,r](-2mm,0pt)
			\JoinCoffins \result[\aaa-B,\aaa-r] \ccc     [B,l](80pt,14pc)
			\JoinCoffins \result[\bbb-l,\ccc-B] \fff     [t,r](-2mm,0pt)
			\JoinCoffins \result[\fff-b,\fff-r] \ruleiii [b,l](2mm,0pt)
			\JoinCoffins \result[\ccc-r,\fff-l] \eee     [B,r](50pt,-50pt)
			\JoinCoffins \result[\eee-T,\eee-r] \ddd     [B,r](0pt,6pc)
			\TypesetCoffin \result
		\end{figure*}
	\end{titlepage} %end
	\newgeometry{a4paper,scale=0.8,centering}
	\frontmatter
	\chapter{Preface}
	Under Construction.
	\tableofcontents
	
	\mainmatter
	\pagestyle{fancy}
	\fancyhf{}
	\cfoot{\thepage}
	\fancyhead[LO]{\slshape \leftmark}
	\fancyhead[RE]{\slshape Linear Algebra}
	\fancyhead[LE,RO]{\slshape \rightmark}
	\renewcommand{\headrulewidth}{0pt}
	\pagenumbering{arabic}
	
	\chapter{Introduction}
		\section{What is Algebra? \& The structure of this note}
			Generally speaking, Algebra is a subject about operations. However, to discuss Algebra in a more detailed way, we have to first talk about some history of Algebra.
			\vspace*{-4mm}
			\begin{table}[H]
				\centering
				\resizebox{\textwidth}{!}{
					\begin{tabular}{lcl}
						\hline \\[-1.75ex]
						& 815 AD         & The first book of Algebra appeared.                                 \\
						\begin{tabular}[c]{@{}l@{}}Era of Elementary Algebra\\ Focusing on:\\ Operations, $\mathbb{C}$, Solving equations%, Properties of roots
						\end{tabular} & $\Bigg\updownarrow$ & 1637: Fermat's Last Theorem                                         \\
						& 1832           &                                                                     \\
						\begin{tabular}[c]{@{}l@{}}Era of Modern Algebra\\ Focusing on:\\ Mathematical Structures and their Morphisms\,
						\end{tabular}  & $\Bigg\updownarrow$ & 1994: Fermat's Last Theorem proved by Wiles\\
						& Now            & \\
						\hline
					\end{tabular}
				}
			\end{table}
			\vspace*{-4mm}
			\noindent To help further understanding, the map of this note is now given.\\
			\textbf{ Part 1: Linear Algebra}
			\begin{table}[H]
				\ifx\du\undefined
				\newlength{\du}
				\fi
				\setlength{\du}{14.220\unitlength}
				\begin{tikzpicture}
				\pgftransformxscale{1.000000}
				\pgftransformyscale{-1.000000}
				\definecolor{dialinecolor}{rgb}{0.000000, 0.000000, 0.000000}
				\pgfsetstrokecolor{dialinecolor}
				\definecolor{dialinecolor}{rgb}{1.000000, 1.000000, 1.000000}
				\pgfsetfillcolor{dialinecolor}
				\pgfsetlinewidth{0.100000\du}
				\pgfsetdash{}{0pt}
				\pgfsetdash{}{0pt}
				\pgfsetmiterjoin
				\definecolor{dialinecolor}{rgb}{1.000000, 1.000000, 1.000000}
				\pgfsetfillcolor{dialinecolor}
				\fill (-4.000000\du,-9.000000\du)--(-4.000000\du,-7.000000\du)--(4.000000\du,-7.000000\du)--(4.000000\du,-9.000000\du)--cycle;
				\definecolor{dialinecolor}{rgb}{0.000000, 0.000000, 0.000000}
				\pgfsetstrokecolor{dialinecolor}
				\draw (-4.000000\du,-9.000000\du)--(-4.000000\du,-7.000000\du)--(4.000000\du,-7.000000\du)--(4.000000\du,-9.000000\du)--cycle;
				% setfont left to latex
				\definecolor{dialinecolor}{rgb}{0.000000, 0.000000, 0.000000}
				\pgfsetstrokecolor{dialinecolor}
				\node[anchor=west] at (0.000000\du,-8.000000\du){};
				% setfont left to latex
				\definecolor{dialinecolor}{rgb}{0.000000, 0.000000, 0.000000}
				\pgfsetstrokecolor{dialinecolor}
				\node at (0.000000\du,-7.9\du){Linear Equation System};
				\pgfsetlinewidth{0.100000\du}
				\pgfsetdash{}{0pt}
				\pgfsetdash{}{0pt}
				\pgfsetmiterjoin
				\definecolor{dialinecolor}{rgb}{1.000000, 1.000000, 1.000000}
				\pgfsetfillcolor{dialinecolor}
				\fill (-15.000000\du,-5.000000\du)--(-15.000000\du,-3.000000\du)--(-7.000000\du,-3.000000\du)--(-7.000000\du,-5.000000\du)--cycle;
				\definecolor{dialinecolor}{rgb}{0.000000, 0.000000, 0.000000}
				\pgfsetstrokecolor{dialinecolor}
				\draw (-15.000000\du,-5.000000\du)--(-15.000000\du,-3.000000\du)--(-7.000000\du,-3.000000\du)--(-7.000000\du,-5.000000\du)--cycle;
				\pgfsetlinewidth{0.100000\du}
				\pgfsetdash{}{0pt}
				\pgfsetdash{}{0pt}
				\pgfsetmiterjoin
				\definecolor{dialinecolor}{rgb}{1.000000, 1.000000, 1.000000}
				\pgfsetfillcolor{dialinecolor}
				\fill (-4.000000\du,-5.000000\du)--(-4.000000\du,-3.000000\du)--(4.000000\du,-3.000000\du)--(4.000000\du,-5.000000\du)--cycle;
				\definecolor{dialinecolor}{rgb}{0.000000, 0.000000, 0.000000}
				\pgfsetstrokecolor{dialinecolor}
				\draw (-4.000000\du,-5.000000\du)--(-4.000000\du,-3.000000\du)--(4.000000\du,-3.000000\du)--(4.000000\du,-5.000000\du)--cycle;
				\pgfsetlinewidth{0.100000\du}
				\pgfsetdash{}{0pt}
				\pgfsetdash{}{0pt}
				\pgfsetmiterjoin
				\definecolor{dialinecolor}{rgb}{1.000000, 1.000000, 1.000000}
				\pgfsetfillcolor{dialinecolor}
				\fill (7.000000\du,-5.000000\du)--(7.000000\du,-3.000000\du)--(15.000000\du,-3.000000\du)--(15.000000\du,-5.000000\du)--cycle;
				\definecolor{dialinecolor}{rgb}{0.000000, 0.000000, 0.000000}
				\pgfsetstrokecolor{dialinecolor}
				\draw (7.000000\du,-5.000000\du)--(7.000000\du,-3.000000\du)--(15.000000\du,-3.000000\du)--(15.000000\du,-5.000000\du)--cycle;
				% setfont left to latex
				\definecolor{dialinecolor}{rgb}{0.000000, 0.000000, 0.000000}
				\pgfsetstrokecolor{dialinecolor}
				\node at (-11.000000\du,-3.90\du){Vector Space};
				% setfont left to latex
				\definecolor{dialinecolor}{rgb}{0.000000, 0.000000, 0.000000}
				\pgfsetstrokecolor{dialinecolor}
				\node at (0.000000\du,-3.975\du){Determinant};
				% setfont left to latex
				\definecolor{dialinecolor}{rgb}{0.000000, 0.000000, 0.000000}
				\pgfsetstrokecolor{dialinecolor}
				\node at (11.000000\du,-3.95\du){Matrix};
				\pgfsetlinewidth{0.100000\du}
				\pgfsetdash{}{0pt}
				\pgfsetdash{}{0pt}
				\pgfsetbuttcap
				{
					\definecolor{dialinecolor}{rgb}{0.000000, 0.000000, 0.000000}
					\pgfsetfillcolor{dialinecolor}
					% was here!!!
					\pgfsetarrowsend{latex}
					\definecolor{dialinecolor}{rgb}{0.000000, 0.000000, 0.000000}
					\pgfsetstrokecolor{dialinecolor}
					\draw (0.000000\du,-7.000000\du)--(0.000000\du,-5.000000\du);
				}
				\pgfsetlinewidth{0.100000\du}
				\pgfsetdash{}{0pt}
				\pgfsetdash{}{0pt}
				\pgfsetbuttcap
				{
					\definecolor{dialinecolor}{rgb}{0.000000, 0.000000, 0.000000}
					\pgfsetfillcolor{dialinecolor}
					% was here!!!
					\pgfsetarrowsstart{latex}
					\pgfsetarrowsend{latex}
					\definecolor{dialinecolor}{rgb}{0.000000, 0.000000, 0.000000}
					\pgfsetstrokecolor{dialinecolor}
					\draw (4.000000\du,-4.000000\du)--(7.000000\du,-4.000000\du);
				}
				\pgfsetlinewidth{0.100000\du}
				\pgfsetdash{}{0pt}
				\pgfsetdash{}{0pt}
				\pgfsetbuttcap
				{
					\definecolor{dialinecolor}{rgb}{0.000000, 0.000000, 0.000000}
					\pgfsetfillcolor{dialinecolor}
					% was here!!!
					\definecolor{dialinecolor}{rgb}{0.000000, 0.000000, 0.000000}
					\pgfsetstrokecolor{dialinecolor}
					\draw (-11.000000\du,-6.000000\du)--(11.000000\du,-6.000000\du);
				}
				\pgfsetlinewidth{0.100000\du}
				\pgfsetdash{}{0pt}
				\pgfsetdash{}{0pt}
				\pgfsetbuttcap
				{
					\definecolor{dialinecolor}{rgb}{0.000000, 0.000000, 0.000000}
					\pgfsetfillcolor{dialinecolor}
					% was here!!!
					\pgfsetarrowsend{latex}
					\definecolor{dialinecolor}{rgb}{0.000000, 0.000000, 0.000000}
					\pgfsetstrokecolor{dialinecolor}
					\draw (-11.000000\du,-6.000000\du)--(-11.000000\du,-5.000000\du);
				}
				\pgfsetlinewidth{0.100000\du}
				\pgfsetdash{}{0pt}
				\pgfsetdash{}{0pt}
				\pgfsetbuttcap
				{
					\definecolor{dialinecolor}{rgb}{0.000000, 0.000000, 0.000000}
					\pgfsetfillcolor{dialinecolor}
					% was here!!!
					\pgfsetarrowsstart{latex}
					\definecolor{dialinecolor}{rgb}{0.000000, 0.000000, 0.000000}
					\pgfsetstrokecolor{dialinecolor}
					\draw (11.000000\du,-5.000000\du)--(11.000000\du,-6.000000\du);
				}
				\pgfsetlinewidth{0.100000\du}
				\pgfsetdash{}{0pt}
				\pgfsetdash{}{0pt}
				\pgfsetmiterjoin
				\definecolor{dialinecolor}{rgb}{1.000000, 1.000000, 1.000000}
				\pgfsetfillcolor{dialinecolor}
				\fill (-15.000000\du,-1.000000\du)--(-15.000000\du,1.000000\du)--(-7.000000\du,1.000000\du)--(-7.000000\du,-1.000000\du)--cycle;
				\definecolor{dialinecolor}{rgb}{0.000000, 0.000000, 0.000000}
				\pgfsetstrokecolor{dialinecolor}
				\draw (-15.000000\du,-1.000000\du)--(-15.000000\du,1.000000\du)--(-7.000000\du,1.000000\du)--(-7.000000\du,-1.000000\du)--cycle;
				\pgfsetlinewidth{0.100000\du}
				\pgfsetdash{}{0pt}
				\pgfsetdash{}{0pt}
				\pgfsetmiterjoin
				\definecolor{dialinecolor}{rgb}{1.000000, 1.000000, 1.000000}
				\pgfsetfillcolor{dialinecolor}
				\fill (7.000000\du,-1.000000\du)--(7.000000\du,1.000000\du)--(15.000000\du,1.000000\du)--(15.000000\du,-1.000000\du)--cycle;
				\definecolor{dialinecolor}{rgb}{0.000000, 0.000000, 0.000000}
				\pgfsetstrokecolor{dialinecolor}
				\draw (7.000000\du,-1.000000\du)--(7.000000\du,1.000000\du)--(15.000000\du,1.000000\du)--(15.000000\du,-1.000000\du)--cycle;
				% setfont left to latex
				\definecolor{dialinecolor}{rgb}{0.000000, 0.000000, 0.000000}
				\pgfsetstrokecolor{dialinecolor}
				\node at (-11.000000\du,0.10\du){Linear Space};
				% setfont left to latex
				\definecolor{dialinecolor}{rgb}{0.000000, 0.000000, 0.000000}
				\pgfsetstrokecolor{dialinecolor}
				\node[anchor=west] at (11.000000\du,1.000000\du){};
				% setfont left to latex
				\definecolor{dialinecolor}{rgb}{0.000000, 0.000000, 0.000000}
				\pgfsetstrokecolor{dialinecolor}
				\node at (11.000000\du,0.06\du){Linear Transformation};
				\pgfsetlinewidth{0.100000\du}
				\pgfsetdash{}{0pt}
				\pgfsetdash{}{0pt}
				\pgfsetbuttcap
				{
					\definecolor{dialinecolor}{rgb}{0.000000, 0.000000, 0.000000}
					\pgfsetfillcolor{dialinecolor}
					% was here!!!
					\pgfsetarrowsend{latex}
					\definecolor{dialinecolor}{rgb}{0.000000, 0.000000, 0.000000}
					\pgfsetstrokecolor{dialinecolor}
					\draw (-11.000000\du,-3.000000\du)--(-11.000000\du,-1.000000\du);
				}
				\pgfsetlinewidth{0.100000\du}
				\pgfsetdash{}{0pt}
				\pgfsetdash{}{0pt}
				\pgfsetbuttcap
				{
					\definecolor{dialinecolor}{rgb}{0.000000, 0.000000, 0.000000}
					\pgfsetfillcolor{dialinecolor}
					% was here!!!
					\pgfsetarrowsend{latex}
					\definecolor{dialinecolor}{rgb}{0.000000, 0.000000, 0.000000}
					\pgfsetstrokecolor{dialinecolor}
					\draw (11.000000\du,-3.000000\du)--(11.000000\du,-1.000000\du);
				}
				\pgfsetlinewidth{0.100000\du}
				\pgfsetdash{}{0pt}
				\pgfsetdash{}{0pt}
				\pgfsetbuttcap
				{
					\definecolor{dialinecolor}{rgb}{0.000000, 0.000000, 0.000000}
					\pgfsetfillcolor{dialinecolor}
					% was here!!!
					\definecolor{dialinecolor}{rgb}{0.000000, 0.000000, 0.000000}
					\pgfsetstrokecolor{dialinecolor}
					\draw (-7.000000\du,0.000000\du)--(7.000000\du,0.000000\du);
				}
				\pgfsetlinewidth{0.100000\du}
				\pgfsetdash{}{0pt}
				\pgfsetdash{}{0pt}
				\pgfsetmiterjoin
				\definecolor{dialinecolor}{rgb}{1.000000, 1.000000, 1.000000}
				\pgfsetfillcolor{dialinecolor}
				\fill (-15.000000\du,2.000000\du)--(-15.000000\du,6.000000\du)--(-7.000000\du,6.000000\du)--(-7.000000\du,2.000000\du)--cycle;
				\definecolor{dialinecolor}{rgb}{0.000000, 0.000000, 0.000000}
				\pgfsetstrokecolor{dialinecolor}
				\draw (-15.000000\du,2.000000\du)--(-15.000000\du,6.000000\du)--(-7.000000\du,6.000000\du)--(-7.000000\du,2.000000\du)--cycle;
				% setfont left to latex
				\definecolor{dialinecolor}{rgb}{0.000000, 0.000000, 0.000000}
				\pgfsetstrokecolor{dialinecolor}
				\node at (-11.000000\du,3.2\du){Bilinear Function};
				% setfont left to latex
				\definecolor{dialinecolor}{rgb}{0.000000, 0.000000, 0.000000}
				\pgfsetstrokecolor{dialinecolor}
				\node at (-11.000000\du,4.03\du){and};
				% setfont left to latex
				\definecolor{dialinecolor}{rgb}{0.000000, 0.000000, 0.000000}
				\pgfsetstrokecolor{dialinecolor}
				\node at (-11.000000\du,4.9\du){Quadratic Form};
				\pgfsetlinewidth{0.100000\du}
				\pgfsetdash{}{0pt}
				\pgfsetdash{}{0pt}
				\pgfsetmiterjoin
				\definecolor{dialinecolor}{rgb}{1.000000, 1.000000, 1.000000}
				\pgfsetfillcolor{dialinecolor}
				\fill (7.000000\du,3.000000\du)--(7.000000\du,5.000000\du)--(15.000000\du,5.000000\du)--(15.000000\du,3.000000\du)--cycle;
				\definecolor{dialinecolor}{rgb}{0.000000, 0.000000, 0.000000}
				\pgfsetstrokecolor{dialinecolor}
				\draw (7.000000\du,3.000000\du)--(7.000000\du,5.000000\du)--(15.000000\du,5.000000\du)--(15.000000\du,3.000000\du)--cycle;
				\pgfsetlinewidth{0.100000\du}
				\pgfsetdash{}{0pt}
				\pgfsetdash{}{0pt}
				\pgfsetbuttcap
				{
					\definecolor{dialinecolor}{rgb}{0.000000, 0.000000, 0.000000}
					\pgfsetfillcolor{dialinecolor}
					% was here!!!
					\pgfsetarrowsend{latex}
					\definecolor{dialinecolor}{rgb}{0.000000, 0.000000, 0.000000}
					\pgfsetstrokecolor{dialinecolor}
					\draw (-11.000000\du,1.000000\du)--(-11.000000\du,2.000000\du);
				}
				\pgfsetlinewidth{0.100000\du}
				\pgfsetdash{}{0pt}
				\pgfsetdash{}{0pt}
				\pgfsetbuttcap
				{
					\definecolor{dialinecolor}{rgb}{0.000000, 0.000000, 0.000000}
					\pgfsetfillcolor{dialinecolor}
					% was here!!!
					\pgfsetarrowsend{latex}
					\definecolor{dialinecolor}{rgb}{0.000000, 0.000000, 0.000000}
					\pgfsetstrokecolor{dialinecolor}
					\draw (11.000000\du,1.000000\du)--(11.000000\du,3.000000\du);
				}
				% setfont left to latex
				\definecolor{dialinecolor}{rgb}{0.000000, 0.000000, 0.000000}
				\pgfsetstrokecolor{dialinecolor}
				\node at (11.000000\du,4.075\du){Jordan Normal Form};
				\pgfsetlinewidth{0.100000\du}
				\pgfsetdash{}{0pt}
				\pgfsetdash{}{0pt}
				\pgfsetbuttcap
				{
					\definecolor{dialinecolor}{rgb}{0.000000, 0.000000, 0.000000}
					\pgfsetfillcolor{dialinecolor}
					% was here!!!
					\definecolor{dialinecolor}{rgb}{0.000000, 0.000000, 0.000000}
					\pgfsetstrokecolor{dialinecolor}
					\draw (-11.000000\du,6.000000\du)--(-11.000000\du,7.000000\du);
				}
				\pgfsetlinewidth{0.100000\du}
				\pgfsetdash{}{0pt}
				\pgfsetdash{}{0pt}
				\pgfsetbuttcap
				{
					\definecolor{dialinecolor}{rgb}{0.000000, 0.000000, 0.000000}
					\pgfsetfillcolor{dialinecolor}
					% was here!!!
					\definecolor{dialinecolor}{rgb}{0.000000, 0.000000, 0.000000}
					\pgfsetstrokecolor{dialinecolor}
					\draw (11.000000\du,5.000000\du)--(11.000000\du,7.000000\du);
				}
				\pgfsetlinewidth{0.100000\du}
				\pgfsetdash{}{0pt}
				\pgfsetdash{}{0pt}
				\pgfsetbuttcap
				{
					\definecolor{dialinecolor}{rgb}{0.000000, 0.000000, 0.000000}
					\pgfsetfillcolor{dialinecolor}
					% was here!!!
					\definecolor{dialinecolor}{rgb}{0.000000, 0.000000, 0.000000}
					\pgfsetstrokecolor{dialinecolor}
					\draw (-11.000000\du,7.000000\du)--(11.000000\du,7.000000\du);
				}
				\pgfsetlinewidth{0.100000\du}
				\pgfsetdash{}{0pt}
				\pgfsetdash{}{0pt}
				\pgfsetmiterjoin
				\definecolor{dialinecolor}{rgb}{1.000000, 1.000000, 1.000000}
				\pgfsetfillcolor{dialinecolor}
				\fill (-7.000000\du,9.000000\du)--(-7.000000\du,11.000000\du)--(7.000000\du,11.000000\du)--(7.000000\du,9.000000\du)--cycle;
				\definecolor{dialinecolor}{rgb}{0.000000, 0.000000, 0.000000}
				\pgfsetstrokecolor{dialinecolor}
				\draw (-7.000000\du,9.000000\du)--(-7.000000\du,11.000000\du)--(7.000000\du,11.000000\du)--(7.000000\du,9.000000\du)--cycle;
				% setfont left to latex
				\definecolor{dialinecolor}{rgb}{0.000000, 0.000000, 0.000000}
				\pgfsetstrokecolor{dialinecolor}
				\node at (0.000000\du,10.15\du){Euclidean Space and Inner Product Space};
				\pgfsetlinewidth{0.100000\du}
				\pgfsetdash{}{0pt}
				\pgfsetdash{}{0pt}
				\pgfsetbuttcap
				{
					\definecolor{dialinecolor}{rgb}{0.000000, 0.000000, 0.000000}
					\pgfsetfillcolor{dialinecolor}
					% was here!!!
					\pgfsetarrowsend{latex}
					\definecolor{dialinecolor}{rgb}{0.000000, 0.000000, 0.000000}
					\pgfsetstrokecolor{dialinecolor}
					\draw (0.000000\du,7.000000\du)--(0.000000\du,9.000000\du);
				}
				\end{tikzpicture}
			\end{table}
			\noindent\textbf{Part 2: Theory of Polynomials}
			\begin{equation*}
			\mathbb{Z} \xrightarrow{\quad\quad\quad} \mathrm{Polynomial~Ring~in~One~Variable} \xrightarrow{\quad\quad\quad} \mathrm{Polynomial~Ring~in~Several~Variables}
			\end{equation*}
			\textbf{Part 3: Tensor Product and Exterior Algebra}
			\begin{equation*}
			\mathrm{Affine~Space} \xrightarrow{\quad\quad\quad} \mathrm{Projective~Space} \xrightarrow{\quad\quad\quad} \mathrm{Tensor~Product} \xrightarrow{\quad\quad\quad} \mathrm{Exterior~Algebra}
			\end{equation*}
			Finally, how to correctly treat matrix is worth discussing. In peronal view, maxtrix is an important tool but it should not dominant this note. For most of the topic, we should use linear space and linear transformation to understand Algebra.
			\section{Several knowledge as preparation}
				\subsection{$\mathbb{C}$}
					Please consult the note: <under construction>
				\subsection{The field of numbers}
					Here we should clarify the object we are going to study for mathematics require rigorness. Therefore, we need the following definition:
					\begin{definition}[The Field of Numbers]
						~\\Let $K$ be a set and $K\subseteq\mathbb{C}$. If $\exists a\in K\implies a\neq0$, $\forall x,y\implies x\pm y,xy\in K$ and $\forall x,y\in K, y\neq 0\implies x/y\in K$. Then $K$ is a field of numbers.
					\end{definition}
					Several common fields of numbers are $\mathbb{C}$, $\mathbb{R}$ and $\mathbb{Q}$. Please note that the set all integers is not a field but since we use it very often, we have a symbol $\mathbb{Z}$ for it.\\
					Also, please aware that there are much more fields of numbers than just $\mathbb{C}$, $\mathbb{R}$ and $\mathbb{Q}$. For example:
					\begin{lemma}
						The set of all Gaussian rationals $\mathbb{Q}(\im):=\{a+b\im\in\mathbb{C}\mid a,b\in \mathbb{Q}\}$ is a field of number.
					\end{lemma}
					\begin{proof}
						$\forall x=a+b\im,y=c+d\im\in\mathbb{Q}(\im):$
						\begin{itemize}
							\item $x\pm y=(a+b\im)\pm(c+d\im)=(a\pm c)+(b\pm d)\im\in \mathbb{Q}(\im)$
							\item $xy=(a+b\im)(c+d\im)=(ac-bd)+(ad+bc)\im\in \mathbb{Q}(\im)$
							\item $\displaystyle\frac{x}{y}=\frac{a+b\im}{c+d\im}=\frac{ac+bd}{c^2+d^2}+\frac{ac+bd}{c^2+d^2}\;\im\in \mathbb{Q}(\im)$
						\end{itemize}
						Therefore, the set of all Gaussian rationals $\mathbb{Q}(\im):=\{a+b\im\in\mathbb{C}\mid a,b\in \mathbb{Q}\}$ is a field of number.
					\end{proof}
					\begin{lemma}
						Let $K$ be any field of numbers, then $\mathbb{Q}\subseteq K$
					\end{lemma}
					\begin{proof}
						We first prove that $\mathbb{N}\subseteq K$, we use induction. In the base case where $n=0$ and $n=1$. Since $K$ is a field fo numbers, $\exists a\in K$ and $a\neq0$. Then $0=a-a\in K$ and $1=a/a\in K$. Now suppose inductively that $n\in K$, then as to $n+1$, $n+1\in K$. This closes the induction, and thus for all natural number $n$, $n\in K$, \textit{i.e.,} $\mathbb{N}\subseteq K$.\\
						We then demonstrate that $\mathbb{Z}\subseteq K$. $\forall x\in\mathbb{Z}, \exists a,b\in\mathbb{N}\implies x=a-b\in K$. Therefore, $\mathbb{Z}\subseteq K$.\\
						We finally show that $\mathbb{Q}\subseteq K$. $\forall x\in\mathbb{Q}, \exists a,b\in\mathbb{Z}\implies x=a/b\in K$. Therefore, $\mathbb{Q}\subseteq K$.
					\end{proof}
				\subsection{Fundamentals of set theory}
					Please consult the note: \textit{Analysis I---Logic, Sets, $\mathbb{N}$, $\mathbb{Z}$ and $\mathbb{Q}$}.
				\subsection{$\Sigma$ and $\Pi$}
					In order to reduce the unnecessary writings, we here introduce the following symbols:
					\begin{definition}[$\Sigma$ and $\Pi$]
						$\forall i,j\in\mathbb{N}_+$
						\begin{align*}
							\sum_{1\leq i\leq n}\mspace{-9mu}a_i=\sum_{i=1}^{n}a_i&:=a_1+a_2+\dotsb +a_n\\
							\prod_{1\leq i\leq n}\mspace{-9mu}a_i=\prod_{i=1}^{n}a_i&:=a_1a_2\dotsm a_n
						\end{align*}
					\end{definition}
				\begin{lemma}[Properties of $\Sigma$]
					$\forall \lambda\in\mathbb{C}$, $\forall i,j\in\mathbb{N}_+$
					\begin{align*}
						\sum_{i}\lambda a_i&=\lambda\sum_{i}a_1\\
						\sum_{i}(a_i+b_i)&=\sum_{i}a_i+\sum_{i}b_i\\
						\sum_{i}\sum_{i}a_{ij}&=\sum_{j}\sum_{i}a_{ij}
					\end{align*}
				\end{lemma}
				\subsection{Fundamentals of logic}
					Please consult the note: \textit{Analysis I---Logic, Sets, $\mathbb{N}$, $\mathbb{Z}$ and $\mathbb{Q}$}.
\part{Linear Algebra}
	\chapter{Vector Space and Matrix}
	under construction
	\chapter{Determinant}
	under construction
	\chapter{Linear Space \& Linear~Transformation}
	under construction
	\chapter{Bilinear~Function \& Quadratic~Form}
	under construction
\part{Polynomial Theory}

\part{Tensor Product \& Exterior~Algebra}


























\end{document}